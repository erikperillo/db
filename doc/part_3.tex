\documentclass[11pt]{article}
\usepackage{graphicx}
\usepackage{float}
\usepackage{amsmath}
\usepackage{amsfonts}
\usepackage[brazilian]{babel}
\usepackage[utf8]{inputenc}
\usepackage[T1]{fontenc}

\newcommand{\tit}[1]{\textit{#1}}
\newcommand{\tbf}[1]{\textbf{#1}}
\newcommand{\ttt}[1]{\texttt{#1}}

\begin{document}

\title{Sistema de Gestão para fábrica de derivados do leite\\
	``Laticínios Prometheus Ltda''}
\author{
	Matheus Ferreira Tavares Boy 103501\\
	Rafael Ossamu Gouvêa 106905\\
	Felipe Viglioni 106665\\
	Erik Perillo 135582}
\date{}
\maketitle

\newpage

\section{Esquema efetivamente implementado}
\paragraph{}
Observações: houve poucas alterações no esquema da parte 2. 
A relação \tit{gera} foi removida pois 
revelou-se não ser necessária aos serviços que implementamos.
As relações \tit{entrega} e \tit{e\_responsavel\_por} foram especializadas
para suas versões relativas \ttt{leite\_uht, queijo, manteiga} e 
\ttt{transportadora, fornecedor, cliente}, respectivamente. 
A relação \tit{compra} foi criada para \ttt{leite\_uht, queijo, manteiga}.
Houve também uma pequena renomeação de \ttt{leite} para \ttt{leite\_uht}.

\begin{itemize}
	\item \tbf{CLIENTE}(cnpj*, razao\_social) -- 234 tuplas
	\item \tbf{FORNECEDOR}(cnpj*, razao\_social) -- 158 tuplas
	\item \tbf{TRANSPORTADORA}(cnpj*, razao\_social) -- 111 tuplas
	\item \tbf{INSUMO}(id*, lote\_insumo, preco, tipo\_gado\_origem, cnpj, data) 
		-- 9001 tuplas
	\item \tbf{LOTE\_INSUMO}(lote\_insumo*, data\_fab, validade) -- 243 tuplas
	\item \tbf{LEITE\_UHT}(id*, lote\_leite, local, preco\_leite, cnpj, data) 
		-- 3321 tuplas
	\item \tbf{LOTE\_LEITE\_UHT}(lote\_leite*, data\_fab, validade,  qualidade) 
		-- 122 tuplas
	\item \tbf{PRECO\_LEITE\_UHT}(preco\_leite*, margem\_de\_lucro) -- 92 tuplas
	\item \tbf{QUEIJO}(id*, lote\_queijo, local, preco\_queijo, cnpj, data) 
		-- 2343 tuplas
	\item \tbf{LOTE\_QUEIJO}(lote\_queijo*, data\_fab, validade,  qualidade) 
		-- 183 tuplas
	\item \tbf{PRECO\_QUEIJO}(preco\_queijo*, margem\_de\_lucro) 
		-- 39 tuplas
	\item \tbf{MANTEIGA}(id*, lote\_manteiga, local, preco\_manteiga, cnpj, data)
		-- 3337
	\item \tbf{LOTE\_MANTEIGA}(lote\_manteiga*, data\_fab, validade,  qualidade, com\_sal,) -- 108 tuplas
	\item \tbf{PRECO\_MANTEIGA}(preco\_manteiga*, margem\_de\_lucro)
		-- 47 tuplas
	\item \tbf{PESSOA}(cpf*, nome, telefone) -- 643 tuplas
	\item \tbf{ESTOQUE}(local*, capacidade, volume\_atual) -- 108 tuplas
	\item \tbf{E\_RESPONSAVEL\_POR\_TRANSPORTADORA}([cpf, cnpj]*) -- 132 tuplas
	\item \tbf{E\_RESPONSAVEL\_POR\_FORNECEDOR}([cpf, cnpj]*) -- 201 tuplas
	\item \tbf{E\_RESPONSAVEL\_POR\_CLIENTE}([cpf, cnpj]*) -- 310 tuplas
	\item \tbf{ENTREGA\_LEITE\_UHT}(id*, cnpj, data) -- 831 tuplas
	\item \tbf{ENTREGA\_QUEIJO}(id*, cnpj, data) -- 322 tuplas
	\item \tbf{ENTREGA\_MANTEIGA}(id*, cnpj, data) -- 523 tuplas
	\item \tbf{COMPRA\_LEITE\_UHT}(id*, cnpj, data) -- 989 tuplas
	\item \tbf{COMPRA\_QUEIJO}(id*, cnpj, data) -- 560 tuplas
	\item \tbf{COMPRA\_MANTEIGA}(id*, cnpj, data) -- 729 tuplas
\end{itemize}

\section{Lista de consultas}

\begin{itemize}
\item Quantidade de cada produto em estoque:
	\ttt{
	\begin{itemize}
    	\item SELECT COUNT(id) FROM leite\_uht WHERE id NOT IN 
			(SELECT id from compra\_leite\_uht);
    	\item SELECT COUNT(id) FROM queijo WHERE id NOT IN 
			(SELECT id from compra\_queijo);
    	\item SELECT COUNT(id) FROM manteiga WHERE id NOT IN 
			(SELECT id from compra\_manteiga);
	\end{itemize}}

\item Valor total dos produtos em estoque
	\ttt{
	\begin{itemize}
		\item SELECT SUM(preco\_leite\_uht) FROM leite\_uht WHERE id NOT IN
			(SELECT id FROM compra\_leite\_uht);
		\item SELECT SUM(preco\_queijo) FROM queijo WHERE id NOT IN
			(SELECT id FROM compra\_queijo);
		\item SELECT SUM(preco\_manteiga) FROM manteiga WHERE id NOT IN
			(SELECT id FROM compra\_manteiga);
	\end{itemize}}

\item Preço individual de cada produto
	\ttt{
	\begin{itemize}
		\item SELECT id, preco\_leite\_uht FROM leite\_uht;
		\item SELECT id, preco\_queijo FROM queijo;
		\item SELECT id, preco\_manteiga FROM manteiga;
	\end{itemize}}

\item Preço por litro de leite de cada fornecedor
	\ttt{
	\begin{itemize}
    	\item SELECT preco, cnpj FROM insumo;
	\end{itemize}}

\item Produtos que cada cliente comprou
	\ttt{
	\begin{itemize}
		\item SELECT id, a.cnpj, razao\_social from 
			cliente a, (SELECT leite\_uht.id, compra\_leite\_uht.cnpj 
							from leite\_uht, compra\_leite\_uht 
							where leite\_uht.id = compra\_leite\_uht.id) b 
			where a.cnpj = b.cnpj;
		\item SELECT id, a.cnpj, razao\_social from 
			cliente a, (SELECT queijo.id, compra\_queijo.cnpj 
							from queijo, compra\_queijo 
							where queijo.id = compra\_queijo.id) b 
			where a.cnpj = b.cnpj;
		\item SELECT id, a.cnpj, razao\_social from 
			cliente a, (SELECT manteiga.id, compra\_manteiga.cnpj 
							from manteiga, compra\_manteiga 
							where manteiga.id = compra\_manteiga.id) b 
			where a.cnpj = b.cnpj;
	\end{itemize}}

\item Data da compra de cada produto que um cliente adquiriu
	\ttt{
	\begin{itemize}
		\item SELECT id, a.cnpj, razao\_social, data from 
			cliente a, (SELECT leite\_uht.id, compra\_leite\_uht.cnpj, compra\_leite\_uht.data
							from leite\_uht, compra\_leite\_uht 
							where leite\_uht.id = compra\_leite\_uht.id) b 
			where a.cnpj = b.cnpj;
		\item SELECT id, a.cnpj, razao\_social, data from 
			cliente a, (SELECT queijo.id, compra\_queijo.cnpj, compra\_queijo.data 
							from queijo, compra\_queijo 
							where queijo.id = compra\_queijo.id) b 
			where a.cnpj = b.cnpj;
		\item SELECT id, a.cnpj, razao\_social, data from 
			cliente a, (SELECT manteiga.id, compra\_manteiga.cnpj, compra\_manteiga.data 
							from manteiga, compra\_manteiga 
							where manteiga.id = compra\_manteiga.id) b 
			where a.cnpj = b.cnpj;
	\end{itemize}}

\item Data da entrega de cada produto que um cliente adquiriu
	\ttt{
	\begin{itemize}
		\item SELECT id, a.cnpj, razao\_social, data from 
			cliente a, (SELECT leite\_uht.id, entrega\_leite\_uht.cnpj, entrega\_leite\_uht.data
							from leite\_uht, entrega\_leite\_uht 
							where leite\_uht.id = entrega\_leite\_uht.id) b 
			where a.cnpj = b.cnpj;
		\item SELECT id, a.cnpj, razao\_social, data from 
			cliente a, (SELECT queijo.id, entrega\_queijo.cnpj, entrega\_queijo.data
							from queijo, entrega\_queijo 
							where queijo.id = entrega\_queijo.id) b 
			where a.cnpj = b.cnpj;
		\item SELECT id, a.cnpj, razao\_social, data from 
			cliente a, (SELECT manteiga.id, entrega\_manteiga.cnpj, entrega\_manteiga.data
							from manteiga, entrega\_manteiga 
							where manteiga.id = entrega\_manteiga.id) b 
			where a.cnpj = b.cnpj;
	\end{itemize}}
\end{itemize}

\end{document}
